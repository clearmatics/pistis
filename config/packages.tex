%!Tex root = ../compliance.tex

\usepackage[utf8]{inputenc}
\usepackage[english]{babel}
\usepackage{graphicx}
\usepackage{xcolor}
\usepackage{float}

\let\spvec\vec
\let\vec\accentvec
\newcommand\hmmax{0}
\newcommand\bmmax{0}
\DeclareFontFamily{U}{mathx}{\hyphenchar\font45}
\DeclareFontShape{U}{mathx}{m}{n}{<-> mathx10}{}
\DeclareSymbolFont{mathx}{U}{mathx}{m}{n}
\DeclareMathAccent{\widebar}{0}{mathx}{"73}

\let\spvec\vec
\usepackage{amssymb,amsmath}
\let\vec\spvec
%\usepackage{newtxmath,newtxtext}
\usepackage{newtxtext}

\usepackage[T1]{fontenc}
\def\vec#1{\mathchoice{\mbox{\boldmath$\displaystyle#1$}}
{\mbox{\boldmath$\textstyle#1$}} {\mbox{\boldmath$\scriptstyle#1$}}
{\mbox{\boldmath$\scriptscriptstyle#1$}}}
%\usepackage{amsthm} % Theorem package to access a few math utils and QED symbols
\usepackage[normalem]{ulem}
\usepackage[color=yellow!20]{todonotes}
\usepackage{soul}
\usepackage[linktoc=all]{hyperref}
\setcounter{tocdepth}{3} % defines how detailed is ToC
\setcounter{secnumdepth}{3} %defines how deeply sections are numerated
\usepackage{paralist}
\usepackage{mathrsfs}
\usepackage{listings}
\usepackage{lscape} % To turn page in landscape mode
\usepackage{dashbox} % Enables to have dashed boxes
\newcommand\dboxed[1]{\dbox{\ensuremath{#1}}}
\usepackage{dirtytalk} % for quotation
\usepackage{booktabs}
\usepackage{pifont}
% See: https://tex.stackexchange.com/questions/2291/how-do-i-change-the-enumerate-list-format-to-use-letters-instead-of-the-defaul
%\usepackage{enumitem} % For enumerates with letters instead of bullets
%\newcommand{\subscript}[2]{$#1 _ #2$} % Command to work with enumitem

%\usepackage[paperheight=23.5cm,paperwidth=15.5cm,text={12.2cm,19.3cm},centering]{geometry}
%\usepackage[paperheight=23.5cm,paperwidth=16.5cm,text={14.2cm,19.3cm},centering]{geometry}
\usepackage[margin=0.8in,a4paper]{geometry}
%\usepackage{fullpage}
\usepackage[framemethod=TikZ,innerleftmargin=5pt,innerrightmargin=5pt]{mdframed}
\usepackage[listings,skins,breakable]{tcolorbox}
\usepackage{bm}
\usepackage[capitalise]{cleveref}
\usepackage[mathscr]{euscript}
\usepackage[all]{nowidow}

\usepackage{sectsty}
\allsectionsfont{\normalfont\sffamily\bfseries}

%\usepackage{fourier}
\usepackage[T1]{fontenc}

\usepackage[advantage,asymptotics,adversary,complexity,sets,keys,ff,notions,
lambda,primitives,events,operators,probability,logic,mm,
landau]{cryptocode}
